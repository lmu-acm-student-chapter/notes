\documentclass[11pt]{article}
\usepackage[margin=1in]{geometry}

\usepackage{tikz-qtree} % \begin{tikzpicture}[scale=.4, align=center] \Tree [.root leaf leaf leaf leaf] \end{tikzpicture}
\usepackage{fancyvrb} % \begin{Verbatim}

\title{Technical Interview Pro-Tips}
\author{from Gayle Laakman's Cracking the Coding Interview}
\date{\today}

\begin{document}
\maketitle

\section{General Advice: Five Steps}

\begin{enumerate}
\item Ask your interview questions to resolve ambiguity
\item Design an algorithm
\item Write pseudo-code first, but make sure your interviewer knows you will write real code later
\item Write your code, not too slow and not too fast
\item Test your code and carefully fix any mistakes
\end{enumerate}

\section{How to Build an Algorithm: Five Approaches}

\begin{enumerate}
\item Write out specific examples of the problem, and see if you can figure out a general rule.
\item Consider what problems the algorithm is similar to, and figure out if you can modify
      the solution to develop an algorithm for this problem.
\item Change a constraint to simplify the problem. Then try to solve it. Once you have a
      solution for the simplified problem, try to generalize it to the original problem.
\item Solve the algorithm first for a base case (e.g, just one element). Then try to solve it
      for elements one and two, and then for two and three, and then generalize.
\item Simply run through a list of data structures and try to apply each one (hacky, but this
      actually works).
\end{enumerate}

\end{document}

